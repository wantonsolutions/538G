\documentclass{article}
\usepackage{pseudocode}
\usepackage{mathtools}
\usepackage{IEEEtrantools}

\author{Stewart Grant 23539133}
\title{CPSC 538G Assignment \#2}
\begin{document}

\maketitle
\tableofcontents
\newpage

\section{Question 1}
\subsection{a}
\textbf{(20 points) Show, by induction on the structure of the graph, that for each vertex $c$, $B\rightarrow q(c)$}

First we define a truth table for $B \rightarrow q(c)$ \\

\begin{tabular}{ c | c | c }
\hline
\textbf{$B$} & \textbf{$q(c)$} & \textbf{$B \rightarrow q(c)$} \\
\hline
\hline
T & T & T \\ \hline
T & F & F \\ \hline
F & T & T \\ \hline
F & F & T \\ \hline
\end{tabular}

\noindent\textbf{(i) $c \in $ root $A$}

By definition $q(c) = true$. In the case where $B = true$ we have
$true \rightarrow true$, in the case where $B$ is false we have $false
\rightarrow true$ which holds. Therefore, $B \rightarrow q(c)$ if $ c
\in $ root $A$.\\


\noindent\textbf{(ii) $c \in $ root $B$}

By definition $q(c) = c$. If $B = true$, then $B \rightarrow q(c)$
becomes $true \rightarrow q(c)$ where $c \in B$ which reduces to $true
\rightarrow true$. If $B = false$, $q(c) \in [true,false]$, so $false
\rightarrow [true,false]$ which holds. Therefore $B \rightarrow q(c)$
if $c \in$ root $B$.\\

\noindent\textbf{(iii) $q(c) = q(c_1) \wedge q(c_2)$}

If $c_1, c_2 \in A$, $q(c) == true$, see \textbf{(i)}. If $c_1, c_2
\in B$, $q(c) = c_1 \wedge c_2$ and $B \rightarrow c_1 \wedge c_2$,
see \textbf{(ii)}. If $c_1 \in A \wedge c_2 \in B$ then $B \rightarrow
true \wedge c_2$. Which reduces to $B \rightarrow c_2$, see
\textbf{(ii)}\\


\subsection{b}
%
\textbf{(20 points) Show, by induction on the structure of the graph,
that for each vertex $c$, $(p(c) \wedge q(c) \rightarrow c$} \\
%
First we define a truth table for $(p(c) \wedge q(c) \rightarrow c$;
taking note that only the second case leads to a contradition.\\
%
\begin{tabular}{ c | c | c | c }
\hline
\textbf{$p(c)$} & \textbf{$q(c)$} & \textbf{$c$} &\textbf{$(p(c) \wedge q(c) \rightarrow c$} \\
\hline
\hline
T & T & T & T \\ \hline
T & T & F & F \\ \hline
T & F & T & T \\ \hline
T & F & F & T \\ \hline
F & T & T & T \\ \hline
F & T & F & T \\ \hline
F & F & T & T \\ \hline
F & F & F & T \\ \hline
\end{tabular}


\noindent\textbf{(i) $c \in $ root $A$}

If $c \in $ root $A$, $q(c) = T$ and $p(c) = g(c)$. This simplifies
the formulat to $T \wedge g(c) \rightarrow c$. Where $g(c) \in c$,
$g(c) = true \rightarrow c$. If $g(c) = false$, $c \in [true,false]$
so $false \rightarrow [true,false]$ which holds. Therefore $q(c)
\wedge p(c) \rightarrow c$ if $c \in $ root $A$.\\


\noindent\textbf{(ii) $c \in $ root $B$}

If $c \in $ root $B$ then $p(c) = true$ and $q(c) = c$. This reduces
to $true \wedge c \rightarrow c$, in tern $true \rightarrow true$,
$false \rightarrow false$. Therefore $q(c) \wedge p(c) \rightarrow c$
if $c \in $ root $B$.\\


\noindent\textbf{(iii) $c$ has parents $c_1$,$c_2$ on variable $v$ which is local to $A$}

If $v$ is local to $A$ then $p(c) = p(c_1) \vee p(c_2)$ and $q(c) =
q(c_1) \wedge q(c_2)$, so we have $(p(c_1) \vee p(c_2)) \wedge (q(c_1)
\wedge q(c_2)) \rightarrow c$. Reducing $p(c)$ to it's globals we get
$(g(c_1) \vee g(c_2)) \wedge (q(c_1) \wedge q(c_2)) \rightarrow c$. If $q(c_1)$,$q(c_2) \in A$ they resolve to true, and from \textbf{a iii} we know .




\end{document}
